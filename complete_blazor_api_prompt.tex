\documentclass[12pt,a4paper]{article}

\usepackage[utf8]{inputenc}
\usepackage[T1]{fontenc}
\usepackage[english]{babel}
\usepackage{geometry}
\usepackage{listings}
\usepackage{xcolor}
\usepackage{booktabs}
\usepackage{multirow}
\usepackage{enumitem}
\usepackage{fancyhdr}
\usepackage{lastpage}

% Code highlighting
\lstset{
    language=C#,
    basicstyle=\ttfamily\footnotesize,
    keywordstyle=\color{blue},
    commentstyle=\color{green!60!black},
    stringstyle=\color{red},
    numbers=left,
    numberstyle=\tiny,
    frame=single,
    breaklines=true,
    captionpos=b
}

% Page setup
\geometry{margin=1in}
\pagestyle{fancy}
\fancyhf{}
\fancyhead[L]{\leftmark}
\fancyhead[R]{\thepage\ of \pageref{LastPage}}
\fancyfoot[C]{\footnotesize Complete Blazor + REST API Implementation Prompt}

\title{Complete Blazor WebAssembly + ASP.NET Core REST API Implementation Prompt}
\author{For AI Assistant Implementation}
\date{\today}

\begin{document}

% Title page
\maketitle
\thispagestyle{empty}
\newpage

\section{Project Overview}

Transform the existing Tunis Motors ASP.NET Core MVC application into a modern microservices architecture with:
\begin{itemize}
    \item \textbf{Backend}: Complete ASP.NET Core Web API with REST endpoints
    \item \textbf{Frontend}: Blazor WebAssembly single-page application
    \item \textbf{Shared}: Common DTOs and models library
\end{itemize}

\section{Current Application Analysis}

\subsection{Existing Features}
\begin{itemize}
    \item User authentication and role-based authorization (Admin, Manager, Seller, Buyer)
    \item Product catalog with car listings (name, brand, price, image, specifications)
    \item Category management with optional images
    \item Shopping cart with real-time updates
    \item Wishlist functionality
    \item Order management and checkout process
    \item User profile management
    \item Admin dashboard with analytics
    \item Approval system for car listings
    \item Message system between users
    \item Responsive Bootstrap 5 design
    \item Font Awesome 6 icons
    \item SQL Server database with Entity Framework Core
\end{itemize}

\subsection{Current Technology Stack}
\begin{itemize}
    \item ASP.NET Core 8.0 MVC
    \item Razor views with server-side rendering
    \item jQuery and AJAX for dynamic updates
    \item Bootstrap 5 + custom CSS
    \item Entity Framework Core with SQL Server
    \item ASP.NET Core Identity for authentication
\end{itemize}

\section{Required Implementation}

\subsection{Phase 1: Project Setup}

Create the following solution structure:

\begin{lstlisting}[caption=Solution Structure]
TunisMotors.Solution/
├── TunisMotors.API/                    # ASP.NET Core Web API
│   ├── Controllers/
│   │   ├── AuthController.cs          # Authentication endpoints
│   │   ├── ProductsController.cs      # Product CRUD operations
│   │   ├── CategoriesController.cs    # Category management
│   │   ├── OrdersController.cs        # Order processing
│   │   ├── CartController.cs          # Shopping cart operations
│   │   ├── WishlistController.cs      # Wishlist management
│   │   ├── UsersController.cs         # User management
│   │   ├── MessagesController.cs      # Messaging system
│   │   └── FilesController.cs         # File upload handling
│   ├── Models/
│   │   ├── Entities/                  # EF Core entities (keep existing)
│   │   └── Configurations/            # Entity configurations
│   ├── Services/
│   │   ├── Interfaces/                # Service interfaces
│   │   ├── Implementations/           # Service implementations
│   │   └── Auth/                      # Authentication services
│   ├── DTOs/
│   │   ├── Requests/                  # Request DTOs
│   │   └── Responses/                 # Response DTOs
│   ├── Data/
│   │   └── AppDbContext.cs           # Database context
│   ├── Middleware/                    # Custom middleware
│   ├── appsettings.json
│   └── Program.cs
├── TunisMotors.Frontend/               # Blazor WebAssembly
│   ├── Pages/
│   │   ├── Index.razor                # Home page
│   │   ├── Products.razor             # Product listings
│   │   ├── ProductDetails.razor       # Product details
│   │   ├── Cart.razor                 # Shopping cart
│   │   ├── Checkout.razor             # Checkout process
│   │   ├── Orders.razor               # Order history
│   │   ├── Profile.razor              # User profile
│   │   ├── Login.razor                # Authentication
│   │   ├── Register.razor             # Registration
│   │   └── Admin/
│   │       ├── Dashboard.razor        # Admin dashboard
│   │       ├── Categories.razor       # Category management
│   │       ├── Users.razor            # User management
│   │       ├── Products.razor         # Product approvals
│   │       └── Analytics.razor        # Analytics view
│   ├── Components/
│   │   ├── Layout/
│   │   │   ├── MainLayout.razor      # Main layout
│   │   │   └── NavMenu.razor         # Navigation menu
│   │   ├── ProductCard.razor         # Product card component
│   │   ├── CategoryFilter.razor      # Category filter
│   │   ├── PriceRangeFilter.razor    # Price filter
│   │   ├── BrandFilter.razor         # Brand filter
│   │   ├── ShoppingCart.razor        # Cart component
│   │   ├── Pagination.razor          # Pagination component
│   │   ├── LoadingSpinner.razor      # Loading indicator
│   │   └── NotificationToast.razor   # Toast notifications
│   ├── Services/
│   │   ├── ApiService.cs             # HTTP client service
│   │   ├── AuthService.cs            # Authentication service
│   │   ├── ProductService.cs         # Product operations
│   │   ├── CartService.cs            # Cart operations
│   │   ├── OrderService.cs           # Order operations
│   │   ├── UserService.cs            # User operations
│   │   ├── CategoryService.cs        # Category operations
│   │   ├── MessageService.cs         # Message operations
│   │   └── NotificationService.cs    # Notification service
│   ├── Shared/
│   │   ├── MainLayout.razor
│   │   ├── RedirectToLogin.razor
│   │   └── App.razor
│   ├── wwwroot/
│   │   ├── index.html
│   │   ├── css/
│   │   └── js/
│   ├── _Imports.razor
│   ├── App.razor
│   └── Program.cs
├── TunisMotors.Shared/                 # Shared components
│   ├── DTOs/
│   │   ├── Auth/
│   │   │   ├── LoginRequest.cs
│   │   │   ├── RegisterRequest.cs
│   │   │   └── AuthResponse.cs
│   │   ├── Products/
│   │   │   ├── ProductDto.cs
│   │   │   ├── CreateProductRequest.cs
│   │   │   ├── UpdateProductRequest.cs
│   │   │   └── ProductFilter.cs
│   │   ├── Categories/
│   │   │   ├── CategoryDto.cs
│   │   │   └── CreateCategoryRequest.cs
│   │   ├── Orders/
│   │   │   ├── OrderDto.cs
│   │   │   ├── CreateOrderRequest.cs
│   │   ├── Cart/
│   │   │   ├── CartItemDto.cs
│   │   │   └── AddToCartRequest.cs
│   │   ├── Users/
│   │   │   ├── UserDto.cs
│   │   │   └── UpdateProfileRequest.cs
│   │   └── Common/
│   │       ├── PagedResult.cs
│   │       ├── ApiResponse.cs
│   │       └── FileUploadRequest.cs
│   ├── Models/
│   │   ├── Enums.cs                   # Application enums
│   │   └── Constants.cs               # Application constants
│   └── Validators/                    # FluentValidation validators
└── TunisMotors.sln
\end{lstlisting}

\subsection{Phase 2: Backend API Implementation}

\subsubsection{2.1 Authentication Setup}

Implement JWT-based authentication in the API:

\begin{lstlisting}[caption=API Authentication Configuration]
// Program.cs - TunisMotors.API
builder.Services.AddAuthentication(JwtBearerDefaults.AuthenticationScheme)
    .AddJwtBearer(options =>
    {
        options.TokenValidationParameters = new TokenValidationParameters
        {
            ValidateIssuer = true,
            ValidateAudience = true,
            ValidateLifetime = true,
            ValidateIssuerSigningKey = true,
            ValidIssuer = builder.Configuration["Jwt:Issuer"],
            ValidAudience = builder.Configuration["Jwt:Audience"],
            IssuerSigningKey = new SymmetricSecurityKey(
                Encoding.UTF8.GetBytes(builder.Configuration["Jwt:Key"]))
        };

        options.Events = new JwtBearerEvents
        {
            OnAuthenticationFailed = context =>
            {
                // Log authentication failures
                return Task.CompletedTask;
            },
            OnTokenValidated = context =>
            {
                // Additional token validation logic
                return Task.CompletedTask;
            }
        };
    });

builder.Services.AddAuthorization(options =>
{
    options.AddPolicy("AdminOnly", policy => policy.RequireRole("Admin"));
    options.AddPolicy("SellerOrAdmin", policy => policy.RequireRole("Seller", "Admin"));
    options.AddPolicy("ManagerOrAdmin", policy => policy.RequireRole("Manager", "Admin"));
});
\end{lstlisting}

\subsubsection{2.2 API Controllers Implementation}

Implement all required API controllers with proper HTTP methods, status codes, and error handling.

\begin{lstlisting}[caption=Products API Controller]
// Controllers/ProductsController.cs
[ApiController]
[Route("api/[controller]")]
public class ProductsController : ControllerBase
{
    private readonly IProductService _productService;
    private readonly IMapper _mapper;

    public ProductsController(IProductService productService, IMapper mapper)
    {
        _productService = productService;
        _mapper = mapper;
    }

    [HttpGet]
    [AllowAnonymous]
    [ProducesResponseType(typeof(PagedResult<ProductDto>), StatusCodes.Status200OK)]
    public async Task<ActionResult<PagedResult<ProductDto>>> GetProducts(
        [FromQuery] ProductFilter filter,
        [FromQuery] int page = 1,
        [FromQuery] int pageSize = 12)
    {
        if (page < 1 || pageSize < 1 || pageSize > 100)
            return BadRequest("Invalid pagination parameters");

        var result = await _productService.GetProductsAsync(filter, page, pageSize);
        return Ok(result);
    }

    [HttpGet("{id}")]
    [AllowAnonymous]
    [ProducesResponseType(typeof(ProductDto), StatusCodes.Status200OK)]
    [ProducesResponseType(StatusCodes.Status404NotFound)]
    public async Task<ActionResult<ProductDto>> GetProduct(int id)
    {
        var product = await _productService.GetProductByIdAsync(id);
        if (product == null)
            return NotFound();

        return Ok(product);
    }

    [HttpPost]
    [Authorize(Roles = "Seller,Admin")]
    [Consumes("multipart/form-data")]
    [ProducesResponseType(typeof(ProductDto), StatusCodes.Status201Created)]
    [ProducesResponseType(StatusCodes.Status400BadRequest)]
    public async Task<ActionResult<ProductDto>> CreateProduct(
        [FromForm] CreateProductRequest request)
    {
        if (!ModelState.IsValid)
            return BadRequest(ModelState);

        try
        {
            var userId = User.FindFirstValue(ClaimTypes.NameIdentifier);
            var product = await _productService.CreateProductAsync(request, userId);
            return CreatedAtAction(nameof(GetProduct), new { id = product.Id }, product);
        }
        catch (ValidationException ex)
        {
            return BadRequest(ex.Message);
        }
        catch (Exception ex)
        {
            // Log the exception
            return StatusCode(500, "An error occurred while creating the product");
        }
    }

    [HttpPut("{id}")]
    [Authorize(Roles = "Seller,Admin")]
    [ProducesResponseType(typeof(ProductDto), StatusCodes.Status200OK)]
    [ProducesResponseType(StatusCodes.Status404NotFound)]
    [ProducesResponseType(StatusCodes.Status403Forbidden)]
    public async Task<ActionResult<ProductDto>> UpdateProduct(
        int id, [FromBody] UpdateProductRequest request)
    {
        if (!ModelState.IsValid)
            return BadRequest(ModelState);

        try
        {
            var userId = User.FindFirstValue(ClaimTypes.NameIdentifier);
            var product = await _productService.UpdateProductAsync(id, request, userId);

            if (product == null)
                return NotFound();

            return Ok(product);
        }
        catch (UnauthorizedAccessException)
        {
            return Forbid();
        }
        catch (Exception ex)
        {
            // Log the exception
            return StatusCode(500, "An error occurred while updating the product");
        }
    }

    [HttpDelete("{id}")]
    [Authorize(Roles = "Seller,Admin")]
    [ProducesResponseType(StatusCodes.Status204NoContent)]
    [ProducesResponseType(StatusCodes.Status404NotFound)]
    [ProducesResponseType(StatusCodes.Status403Forbidden)]
    public async Task<IActionResult> DeleteProduct(int id)
    {
        try
        {
            var userId = User.FindFirstValue(ClaimTypes.NameIdentifier);
            var result = await _productService.DeleteProductAsync(id, userId);

            if (!result)
                return NotFound();

            return NoContent();
        }
        catch (UnauthorizedAccessException)
        {
            return Forbid();
        }
        catch (Exception ex)
        {
            // Log the exception
            return StatusCode(500, "An error occurred while deleting the product");
        }
    }

    [HttpPost("{id}/approve")]
    [Authorize(Roles = "Manager,Admin")]
    [ProducesResponseType(typeof(ProductDto), StatusCodes.Status200OK)]
    [ProducesResponseType(StatusCodes.Status404NotFound)]
    public async Task<ActionResult<ProductDto>> ApproveProduct(int id)
    {
        var product = await _productService.ApproveProductAsync(id);
        if (product == null)
            return NotFound();

        return Ok(product);
    }

    [HttpPost("{id}/reject")]
    [Authorize(Roles = "Manager,Admin")]
    [ProducesResponseType(typeof(ProductDto), StatusCodes.Status200OK)]
    [ProducesResponseType(StatusCodes.Status404NotFound)]
    public async Task<ActionResult<ProductDto>> RejectProduct(int id, [FromBody] string reason)
    {
        var product = await _productService.RejectProductAsync(id, reason);
        if (product == null)
            return NotFound();

        return Ok(product);
    }
}
\end{lstlisting}

\subsubsection{2.3 Authentication Controller}

\begin{lstlisting}[caption=Authentication API Controller]
// Controllers/AuthController.cs
[ApiController]
[Route("api/[controller]")]
public class AuthController : ControllerBase
{
    private readonly IAuthService _authService;
    private readonly IMapper _mapper;

    public AuthController(IAuthService authService, IMapper mapper)
    {
        _authService = authService;
        _mapper = mapper;
    }

    [HttpPost("login")]
    [AllowAnonymous]
    [ProducesResponseType(typeof(AuthResponse), StatusCodes.Status200OK)]
    [ProducesResponseType(StatusCodes.Status400BadRequest)]
    [ProducesResponseType(StatusCodes.Status401Unauthorized)]
    public async Task<ActionResult<AuthResponse>> Login([FromBody] LoginRequest request)
    {
        if (!ModelState.IsValid)
            return BadRequest(ModelState);

        var result = await _authService.LoginAsync(request);
        if (result == null)
            return Unauthorized("Invalid credentials");

        return Ok(result);
    }

    [HttpPost("register")]
    [AllowAnonymous]
    [ProducesResponseType(typeof(AuthResponse), StatusCodes.Status201Created)]
    [ProducesResponseType(StatusCodes.Status400BadRequest)]
    public async Task<ActionResult<AuthResponse>> Register([FromBody] RegisterRequest request)
    {
        if (!ModelState.IsValid)
            return BadRequest(ModelState);

        try
        {
            var result = await _authService.RegisterAsync(request);
            return CreatedAtAction(nameof(Login), result);
        }
        catch (ValidationException ex)
        {
            return BadRequest(ex.Message);
        }
        catch (Exception ex)
        {
            // Log the exception
            return StatusCode(500, "An error occurred during registration");
        }
    }

    [HttpPost("refresh")]
    [AllowAnonymous]
    [ProducesResponseType(typeof(AuthResponse), StatusCodes.Status200OK)]
    [ProducesResponseType(StatusCodes.Status400BadRequest)]
    public async Task<ActionResult<AuthResponse>> RefreshToken([FromBody] RefreshTokenRequest request)
    {
        if (!ModelState.IsValid)
            return BadRequest(ModelState);

        var result = await _authService.RefreshTokenAsync(request);
        if (result == null)
            return BadRequest("Invalid refresh token");

        return Ok(result);
    }

    [HttpPost("logout")]
    [Authorize]
    [ProducesResponseType(StatusCodes.Status204NoContent)]
    public async Task<IActionResult> Logout()
    {
        var userId = User.FindFirstValue(ClaimTypes.NameIdentifier);
        await _authService.LogoutAsync(userId);
        return NoContent();
    }

    [HttpGet("me")]
    [Authorize]
    [ProducesResponseType(typeof(UserDto), StatusCodes.Status200OK)]
    public async Task<ActionResult<UserDto>> GetCurrentUser()
    {
        var userId = User.FindFirstValue(ClaimTypes.NameIdentifier);
        var user = await _authService.GetCurrentUserAsync(userId);

        if (user == null)
            return NotFound();

        return Ok(user);
    }

    [HttpPut("profile")]
    [Authorize]
    [ProducesResponseType(typeof(UserDto), StatusCodes.Status200OK)]
    [ProducesResponseType(StatusCodes.Status400BadRequest)]
    public async Task<ActionResult<UserDto>> UpdateProfile([FromBody] UpdateProfileRequest request)
    {
        if (!ModelState.IsValid)
            return BadRequest(ModelState);

        try
        {
            var userId = User.FindFirstValue(ClaimTypes.NameIdentifier);
            var user = await _authService.UpdateProfileAsync(userId, request);
            return Ok(user);
        }
        catch (ValidationException ex)
        {
            return BadRequest(ex.Message);
        }
        catch (Exception ex)
        {
            // Log the exception
            return StatusCode(500, "An error occurred while updating profile");
        }
    }
}
\end{lstlisting}

\subsubsection{2.4 Cart and Orders Controllers}

\begin{lstlisting}[caption=Cart API Controller]
// Controllers/CartController.cs
[ApiController]
[Route("api/[controller]")]
[Authorize]
public class CartController : ControllerBase
{
    private readonly ICartService _cartService;

    public CartController(ICartService cartService)
    {
        _cartService = cartService;
    }

    [HttpGet]
    [ProducesResponseType(typeof(IEnumerable<CartItemDto>), StatusCodes.Status200OK)]
    public async Task<ActionResult<IEnumerable<CartItemDto>>> GetCart()
    {
        var userId = User.FindFirstValue(ClaimTypes.NameIdentifier);
        var cartItems = await _cartService.GetCartItemsAsync(userId);
        return Ok(cartItems);
    }

    [HttpPost("add")]
    [ProducesResponseType(typeof(CartItemDto), StatusCodes.Status201Created)]
    [ProducesResponseType(StatusCodes.Status400BadRequest)]
    public async Task<ActionResult<CartItemDto>> AddToCart([FromBody] AddToCartRequest request)
    {
        if (!ModelState.IsValid)
            return BadRequest(ModelState);

        try
        {
            var userId = User.FindFirstValue(ClaimTypes.NameIdentifier);
            var cartItem = await _cartService.AddToCartAsync(userId, request);
            return CreatedAtAction(nameof(GetCart), cartItem);
        }
        catch (ValidationException ex)
        {
            return BadRequest(ex.Message);
        }
        catch (Exception ex)
        {
            // Log the exception
            return StatusCode(500, "An error occurred while adding item to cart");
        }
    }

    [HttpPut("{productId}")]
    [ProducesResponseType(typeof(CartItemDto), StatusCodes.Status200OK)]
    [ProducesResponseType(StatusCodes.Status404NotFound)]
    public async Task<ActionResult<CartItemDto>> UpdateCartItem(
        int productId, [FromBody] UpdateCartItemRequest request)
    {
        if (!ModelState.IsValid)
            return BadRequest(ModelState);

        try
        {
            var userId = User.FindFirstValue(ClaimTypes.NameIdentifier);
            var cartItem = await _cartService.UpdateCartItemAsync(userId, productId, request.Quantity);

            if (cartItem == null)
                return NotFound();

            return Ok(cartItem);
        }
        catch (ValidationException ex)
        {
            return BadRequest(ex.Message);
        }
        catch (Exception ex)
        {
            // Log the exception
            return StatusCode(500, "An error occurred while updating cart item");
        }
    }

    [HttpDelete("{productId}")]
    [ProducesResponseType(StatusCodes.Status204NoContent)]
    [ProducesResponseType(StatusCodes.Status404NotFound)]
    public async Task<IActionResult> RemoveFromCart(int productId)
    {
        var userId = User.FindFirstValue(ClaimTypes.NameIdentifier);
        var result = await _cartService.RemoveFromCartAsync(userId, productId);

        if (!result)
            return NotFound();

        return NoContent();
    }

    [HttpDelete]
    [ProducesResponseType(StatusCodes.Status204NoContent)]
    public async Task<IActionResult> ClearCart()
    {
        var userId = User.FindFirstValue(ClaimTypes.NameIdentifier);
        await _cartService.ClearCartAsync(userId);
        return NoContent();
    }

    [HttpGet("count")]
    [ProducesResponseType(typeof(int), StatusCodes.Status200OK)]
    public async Task<ActionResult<int>> GetCartCount()
    {
        var userId = User.FindFirstValue(ClaimTypes.NameIdentifier);
        var count = await _cartService.GetCartCountAsync(userId);
        return Ok(count);
    }
}
\end{lstlisting}

\subsubsection{2.5 File Upload Controller}

\begin{lstlisting}[caption=File Upload API Controller]
// Controllers/FilesController.cs
[ApiController]
[Route("api/[controller]")]
public class FilesController : ControllerBase
{
    private readonly IWebHostEnvironment _environment;

    public FilesController(IWebHostEnvironment environment)
    {
        _environment = environment;
    }

    [HttpPost("upload")]
    [Authorize]
    [ProducesResponseType(typeof(FileUploadResponse), StatusCodes.Status200OK)]
    [ProducesResponseType(StatusCodes.Status400BadRequest)]
    public async Task<ActionResult<FileUploadResponse>> UploadFile(
        [FromForm] IFormFile file, [FromForm] string folder = "general")
    {
        if (file == null || file.Length == 0)
            return BadRequest("No file uploaded");

        // Validate file type
        var allowedExtensions = new[] { ".jpg", ".jpeg", ".png", ".gif", ".pdf" };
        var extension = Path.GetExtension(file.FileName).ToLower();
        if (!allowedExtensions.Contains(extension))
            return BadRequest("Invalid file type");

        // Validate file size (max 10MB)
        if (file.Length > 10 * 1024 * 1024)
            return BadRequest("File size exceeds 10MB limit");

        try
        {
            // Create folder if it doesn't exist
            var uploadsFolder = Path.Combine(_environment.WebRootPath, "uploads", folder);
            if (!Directory.Exists(uploadsFolder))
                Directory.CreateDirectory(uploadsFolder);

            // Generate unique filename
            var fileName = Guid.NewGuid().ToString() + extension;
            var filePath = Path.Combine(uploadsFolder, fileName);

            // Save file
            using (var stream = new FileStream(filePath, FileMode.Create))
            {
                await file.CopyToAsync(stream);
            }

            var response = new FileUploadResponse
            {
                FileName = fileName,
                OriginalFileName = file.FileName,
                FileSize = file.Length,
                ContentType = file.ContentType,
                Url = $"/uploads/{folder}/{fileName}"
            };

            return Ok(response);
        }
        catch (Exception ex)
        {
            // Log the exception
            return StatusCode(500, "An error occurred while uploading the file");
        }
    }

    [HttpDelete("{folder}/{fileName}")]
    [Authorize]
    [ProducesResponseType(StatusCodes.Status204NoContent)]
    [ProducesResponseType(StatusCodes.Status404NotFound)]
    public IActionResult DeleteFile(string folder, string fileName)
    {
        var filePath = Path.Combine(_environment.WebRootPath, "uploads", folder, fileName);

        if (!System.IO.File.Exists(filePath))
            return NotFound();

        try
        {
            System.IO.File.Delete(filePath);
            return NoContent();
        }
        catch (Exception ex)
        {
            // Log the exception
            return StatusCode(500, "An error occurred while deleting the file");
        }
    }
}
\end{lstlisting}

\subsection{Phase 3: Frontend Blazor Implementation}

\subsubsection{3.1 Program.cs Configuration}

\begin{lstlisting}[caption=Blazor Program.cs]
// Program.cs - TunisMotors.Frontend
using Microsoft.AspNetCore.Components.Web;
using Microsoft.AspNetCore.Components.WebAssembly.Hosting;
using MudBlazor.Services;
using Blazored.LocalStorage;
using TunisMotors.Frontend.Services;

var builder = WebAssemblyHostBuilder.CreateDefault(args);
builder.RootComponents.Add<App>("#app");
builder.RootComponents.Add<HeadOutlet>("head::after");

builder.Services.AddScoped(sp => new HttpClient
{
    BaseAddress = new Uri(builder.HostEnvironment.BaseAddress)
});

// Add MudBlazor services
builder.Services.AddMudServices();

// Add local storage
builder.Services.AddBlazoredLocalStorage();

// Add custom services
builder.Services.AddScoped<IApiService, ApiService>();
builder.Services.AddScoped<IAuthService, AuthService>();
builder.Services.AddScoped<IProductService, ProductService>();
builder.Services.AddScoped<ICartService, CartService>();
builder.Services.AddScoped<IOrderService, OrderService>();
builder.Services.AddScoped<IUserService, UserService>();
builder.Services.AddScoped<ICategoryService, CategoryService>();
builder.Services.AddScoped<IMessageService, MessageService>();
builder.Services.AddScoped<INotificationService, NotificationService>();

// Add AutoMapper
builder.Services.AddAutoMapper(typeof(Program));

await builder.Build().RunAsync();
\end{lstlisting}

\subsubsection{3.2 API Service Implementation}

\begin{lstlisting}[caption=API Service in Blazor]
// Services/ApiService.cs
public class ApiService : IApiService
{
    private readonly HttpClient _httpClient;
    private readonly ILocalStorageService _localStorage;
    private readonly AuthenticationStateProvider _authStateProvider;
    private readonly ILogger<ApiService> _logger;

    public ApiService(
        HttpClient httpClient,
        ILocalStorageService localStorage,
        AuthenticationStateProvider authStateProvider,
        ILogger<ApiService> logger)
    {
        _httpClient = httpClient;
        _localStorage = localStorage;
        _authStateProvider = authStateProvider;
        _logger = logger;

        // Set base address for API
        _httpClient.BaseAddress = new Uri("https://localhost:5001/api/");
        _httpClient.DefaultRequestHeaders.Accept.Add(
            new MediaTypeWithQualityHeaderValue("application/json"));
    }

    public async Task<T> GetAsync<T>(string endpoint, bool requireAuth = true)
    {
        try
        {
            if (requireAuth)
                await SetAuthorizationHeader();

            var response = await _httpClient.GetAsync(endpoint);
            await HandleResponse(response);
            return await response.Content.ReadFromJsonAsync<T>();
        }
        catch (Exception ex)
        {
            _logger.LogError(ex, $"Error calling GET {endpoint}");
            throw;
        }
    }

    public async Task<T> PostAsync<T>(string endpoint, object data, bool requireAuth = true)
    {
        try
        {
            if (requireAuth)
                await SetAuthorizationHeader();

            var response = await _httpClient.PostAsJsonAsync(endpoint, data);
            await HandleResponse(response);
            return await response.Content.ReadFromJsonAsync<T>();
        }
        catch (Exception ex)
        {
            _logger.LogError(ex, $"Error calling POST {endpoint}");
            throw;
        }
    }

    public async Task<T> PutAsync<T>(string endpoint, object data, bool requireAuth = true)
    {
        try
        {
            if (requireAuth)
                await SetAuthorizationHeader();

            var response = await _httpClient.PutAsJsonAsync(endpoint, data);
            await HandleResponse(response);
            return await response.Content.ReadFromJsonAsync<T>();
        }
        catch (Exception ex)
        {
            _logger.LogError(ex, $"Error calling PUT {endpoint}");
            throw;
        }
    }

    public async Task DeleteAsync(string endpoint, bool requireAuth = true)
    {
        try
        {
            if (requireAuth)
                await SetAuthorizationHeader();

            var response = await _httpClient.DeleteAsync(endpoint);
            await HandleResponse(response);
        }
        catch (Exception ex)
        {
            _logger.LogError(ex, $"Error calling DELETE {endpoint}");
            throw;
        }
    }

    public async Task<T> PostFileAsync<T>(string endpoint, MultipartFormDataContent content, bool requireAuth = true)
    {
        try
        {
            if (requireAuth)
                await SetAuthorizationHeader();

            var response = await _httpClient.PostAsync(endpoint, content);
            await HandleResponse(response);
            return await response.Content.ReadFromJsonAsync<T>();
        }
        catch (Exception ex)
        {
            _logger.LogError(ex, $"Error calling POST file {endpoint}");
            throw;
        }
    }

    private async Task SetAuthorizationHeader()
    {
        var token = await _localStorage.GetItemAsync<string>("authToken");
        if (!string.IsNullOrEmpty(token))
        {
            _httpClient.DefaultRequestHeaders.Authorization =
                new AuthenticationHeaderValue("Bearer", token);
        }
    }

    private async Task HandleResponse(HttpResponseMessage response)
    {
        if (!response.IsSuccessStatusCode)
        {
            var errorContent = await response.Content.ReadAsStringAsync();
            throw new HttpRequestException($"API call failed: {response.StatusCode} - {errorContent}");
        }
    }
}
\end{lstlisting}

\subsubsection{3.3 Authentication Service}

\begin{lstlisting}[caption=Authentication Service in Blazor]
// Services/AuthService.cs
public class AuthService : IAuthService
{
    private readonly IApiService _apiService;
    private readonly ILocalStorageService _localStorage;
    private readonly AuthenticationStateProvider _authStateProvider;
    private readonly NavigationManager _navigation;
    private readonly ISnackbar _snackbar;

    public AuthService(
        IApiService apiService,
        ILocalStorageService localStorage,
        AuthenticationStateProvider authStateProvider,
        NavigationManager navigation,
        ISnackbar snackbar)
    {
        _apiService = apiService;
        _localStorage = localStorage;
        _authStateProvider = authStateProvider;
        _navigation = navigation;
        _snackbar = snackbar;
    }

    public async Task<bool> LoginAsync(LoginRequest request)
    {
        try
        {
            var response = await _apiService.PostAsync<AuthResponse>("auth/login", request, false);

            await _localStorage.SetItemAsync("authToken", response.Token);
            await _localStorage.SetItemAsync("refreshToken", response.RefreshToken);
            await _localStorage.SetItemAsync("user", response.User);

            ((CustomAuthStateProvider)_authStateProvider).NotifyUserAuthentication(response.User);

            _snackbar.Add("Login successful!", Severity.Success);
            _navigation.NavigateTo("/");

            return true;
        }
        catch (Exception ex)
        {
            _snackbar.Add($"Login failed: {ex.Message}", Severity.Error);
            return false;
        }
    }

    public async Task<bool> RegisterAsync(RegisterRequest request)
    {
        try
        {
            var response = await _apiService.PostAsync<AuthResponse>("auth/register", request, false);

            await _localStorage.SetItemAsync("authToken", response.Token);
            await _localStorage.SetItemAsync("refreshToken", response.RefreshToken);
            await _localStorage.SetItemAsync("user", response.User);

            ((CustomAuthStateProvider)_authStateProvider).NotifyUserAuthentication(response.User);

            _snackbar.Add("Registration successful!", Severity.Success);
            _navigation.NavigateTo("/");

            return true;
        }
        catch (Exception ex)
        {
            _snackbar.Add($"Registration failed: {ex.Message}", Severity.Error);
            return false;
        }
    }

    public async Task LogoutAsync()
    {
        try
        {
            await _apiService.PostAsync<object>("auth/logout", null);
        }
        catch
        {
            // Ignore logout API errors
        }

        await _localStorage.RemoveItemAsync("authToken");
        await _localStorage.RemoveItemAsync("refreshToken");
        await _localStorage.RemoveItemAsync("user");

        ((CustomAuthStateProvider)_authStateProvider).NotifyUserLogout();

        _snackbar.Add("Logged out successfully", Severity.Info);
        _navigation.NavigateTo("/login");
    }

    public async Task<UserDto> GetCurrentUserAsync()
    {
        return await _localStorage.GetItemAsync<UserDto>("user");
    }

    public async Task<bool> IsAuthenticatedAsync()
    {
        var token = await _localStorage.GetItemAsync<string>("authToken");
        return !string.IsNullOrEmpty(token);
    }

    public async Task<string> GetTokenAsync()
    {
        return await _localStorage.GetItemAsync<string>("authToken");
    }

    public async Task RefreshTokenAsync()
    {
        try
        {
            var refreshToken = await _localStorage.GetItemAsync<string>("refreshToken");
            if (string.IsNullOrEmpty(refreshToken))
                throw new Exception("No refresh token available");

            var request = new RefreshTokenRequest { RefreshToken = refreshToken };
            var response = await _apiService.PostAsync<AuthResponse>("auth/refresh", request, false);

            await _localStorage.SetItemAsync("authToken", response.Token);
            await _localStorage.SetItemAsync("refreshToken", response.RefreshToken);
        }
        catch (Exception ex)
        {
            // If refresh fails, logout user
            await LogoutAsync();
            throw new Exception($"Token refresh failed: {ex.Message}");
        }
    }
}
\end{lstlisting}

\subsubsection{3.4 Product Service}

\begin{lstlisting}[caption=Product Service in Blazor]
// Services/ProductService.cs
public class ProductService : IProductService
{
    private readonly IApiService _apiService;
    private readonly ISnackbar _snackbar;

    public ProductService(IApiService apiService, ISnackbar snackbar)
    {
        _apiService = apiService;
        _snackbar = snackbar;
    }

    public async Task<PagedResult<ProductDto>> GetProductsAsync(
        ProductFilter filter = null, int page = 1, int pageSize = 12)
    {
        var queryParams = new List<string>();

        if (filter != null)
        {
            if (!string.IsNullOrEmpty(filter.SearchTerm))
                queryParams.Add($"searchTerm={Uri.EscapeDataString(filter.SearchTerm)}");

            if (filter.CategoryId.HasValue)
                queryParams.Add($"categoryId={filter.CategoryId}");

            if (filter.MinPrice.HasValue)
                queryParams.Add($"minPrice={filter.MinPrice}");

            if (filter.MaxPrice.HasValue)
                queryParams.Add($"maxPrice={filter.MaxPrice}");

            if (!string.IsNullOrEmpty(filter.Brand))
                queryParams.Add($"brand={Uri.EscapeDataString(filter.Brand)}");

            if (filter.Condition.HasValue)
                queryParams.Add($"condition={filter.Condition}");
        }

        queryParams.Add($"page={page}");
        queryParams.Add($"pageSize={pageSize}");

        var queryString = string.Join("&", queryParams);
        var endpoint = $"products?{queryString}";

        return await _apiService.GetAsync<PagedResult<ProductDto>>(endpoint, false);
    }

    public async Task<ProductDto> GetProductByIdAsync(int id)
    {
        return await _apiService.GetAsync<ProductDto>($"products/{id}", false);
    }

    public async Task<ProductDto> CreateProductAsync(CreateProductRequest request)
    {
        var content = new MultipartFormDataContent();

        // Add form data
        content.Add(new StringContent(request.Name), "Name");
        content.Add(new StringContent(request.Brand), "Brand");
        content.Add(new StringContent(request.Price.ToString()), "Price");
        content.Add(new StringContent(request.Description ?? ""), "Description");
        content.Add(new StringContent(request.Year.ToString()), "Year");
        content.Add(new StringContent(request.Mileage?.ToString() ?? "0"), "Mileage");
        content.Add(new StringContent(request.FuelType), "FuelType");
        content.Add(new StringContent(request.Transmission), "Transmission");
        content.Add(new StringContent(request.Color ?? ""), "Color");
        content.Add(new StringContent(request.Condition.ToString()), "Condition");
        content.Add(new StringContent(request.CategoryId.ToString()), "CategoryId");

        // Add specifications
        if (request.Specifications != null)
        {
            content.Add(new StringContent(
                System.Text.Json.JsonSerializer.Serialize(request.Specifications)), "Specifications");
        }

        // Add image file
        if (request.Image != null)
        {
            var fileContent = new StreamContent(request.Image.OpenReadStream());
            fileContent.Headers.ContentType = new MediaTypeHeaderValue(request.Image.ContentType);
            content.Add(fileContent, "Image", request.Image.Name);
        }

        var result = await _apiService.PostFileAsync<ProductDto>("products", content);
        _snackbar.Add("Product created successfully!", Severity.Success);
        return result;
    }

    public async Task<ProductDto> UpdateProductAsync(int id, UpdateProductRequest request)
    {
        var result = await _apiService.PutAsync<ProductDto>($"products/{id}", request);
        _snackbar.Add("Product updated successfully!", Severity.Success);
        return result;
    }

    public async Task<bool> DeleteProductAsync(int id)
    {
        try
        {
            await _apiService.DeleteAsync($"products/{id}");
            _snackbar.Add("Product deleted successfully!", Severity.Success);
            return true;
        }
        catch
        {
            _snackbar.Add("Failed to delete product", Severity.Error);
            return false;
        }
    }

    public async Task<IEnumerable<string>> GetBrandsAsync()
    {
        return await _apiService.GetAsync<IEnumerable<string>>("products/brands", false);
    }

    public async Task<IEnumerable<ProductDto>> GetFeaturedProductsAsync(int count = 8)
    {
        return await _apiService.GetAsync<IEnumerable<ProductDto>>($"products/featured?count={count}", false);
    }

    public async Task<IEnumerable<ProductDto>> GetRelatedProductsAsync(int productId, int count = 4)
    {
        return await _apiService.GetAsync<IEnumerable<ProductDto>>($"products/{productId}/related?count={count}", false);
    }
}
\end{lstlisting}

\subsubsection{3.5 Main Product Listing Page}

\begin{lstlisting}[caption=Products Page in Blazor]
@page "/products"
@inject IProductService ProductService
@inject NavigationManager Navigation
@inject ISnackbar Snackbar
@inject IAuthService AuthService

<PageTitle>Car Listings - Tunis Motors</PageTitle>

<div class="container-fluid mt-4">
    <div class="row">
        <!-- Filters Sidebar -->
        <div class="col-lg-3 mb-4">
            <div class="card shadow">
                <div class="card-header bg-primary text-white">
                    <h5 class="mb-0">
                        <Icon Icon="@Icons.Material.Filled.FilterList" Class="me-2" />
                        Filters
                    </h5>
                </div>
                <div class="card-body">
                    <!-- Search -->
                    <div class="mb-3">
                        <MudTextField @bind-Value="filter.SearchTerm"
                                    Label="Search cars..."
                                    Variant="Variant.Outlined"
                                    Adornment="Adornment.Start"
                                    AdornmentIcon="@Icons.Material.Filled.Search"
                                    FullWidth="true" />
                    </div>

                    <!-- Category Filter -->
                    <CategoryFilter @bind-SelectedCategories="filter.Categories"
                                  @bind-SelectedCategoriesChanged="OnCategoriesChanged" />

                    <!-- Price Range Filter -->
                    <div class="mb-3">
                        <MudTextField @bind-Value="filter.MinPrice"
                                    Label="Min Price (TND)"
                                    Variant="Variant.Outlined"
                                    Adornment="Adornment.Start"
                                    AdornmentText="TND"
                                    FullWidth="true"
                                    InputType="InputType.Number" />
                        <MudTextField @bind-Value="filter.MaxPrice"
                                    Label="Max Price (TND)"
                                    Variant="Variant.Outlined"
                                    Adornment="Adornment.Start"
                                    AdornmentText="TND"
                                    FullWidth="true"
                                    InputType="InputType.Number" />
                    </div>

                    <!-- Brand Filter -->
                    <BrandFilter @bind-SelectedBrands="filter.Brands"
                               @bind-SelectedBrandsChanged="OnBrandsChanged" />

                    <!-- Condition Filter -->
                    <div class="mb-3">
                        <MudSelect @bind-Value="filter.Condition"
                                 Label="Condition"
                                 Variant="Variant.Outlined"
                                 FullWidth="true"
                                 Clearable="true">
                            <MudSelectItem Value="ProductCondition.New">New</MudSelectItem>
                            <MudSelectItem Value="ProductCondition.Used">Used</MudSelectItem>
                        </MudSelect>
                    </div>

                    <!-- Apply Filters Button -->
                    <MudButton Variant="Variant.Filled"
                             Color="Color.Primary"
                             FullWidth="true"
                             StartIcon="@Icons.Material.Filled.FilterAlt"
                             @onclick="ApplyFilters">
                        Apply Filters
                    </MudButton>

                    <!-- Clear Filters Button -->
                    <MudButton Variant="Variant.Outlined"
                             Color="Color.Secondary"
                             FullWidth="true"
                             Class="mt-2"
                             StartIcon="@Icons.Material.Filled.Clear"
                             @onclick="ClearFilters">
                        Clear Filters
                    </MudButton>
                </div>
            </div>
        </div>

        <!-- Products Grid -->
        <div class="col-lg-9">
            <div class="d-flex justify-content-between align-items-center mb-4">
                <div>
                    <h2 class="mb-0">Car Listings</h2>
                    @if (totalCount > 0)
                    {
                        <small class="text-muted">@totalCount cars found</small>
                    }
                </div>

                <!-- View Toggle -->
                <div class="d-flex gap-2">
                    <MudButtonGroup Color="Color.Primary" Variant="Variant.Outlined">
                        <MudButton StartIcon="@Icons.Material.Filled.GridView"
                                 @onclick="() => viewMode = ViewMode.Grid"
                                 Color="viewMode == ViewMode.Grid ? Color.Primary : Color.Default"
                                 Variant="viewMode == ViewMode.Grid ? Variant.Filled : Variant.Outlined">
                            Grid
                        </MudButton>
                        <MudButton StartIcon="@Icons.Material.Filled.List"
                                 @onclick="() => viewMode = ViewMode.List"
                                 Color="viewMode == ViewMode.List ? Color.Primary : Color.Default"
                                 Variant="viewMode == ViewMode.List ? Variant.Filled : Variant.Outlined">
                            List
                        </MudButton>
                    </MudButtonGroup>
                </div>
            </div>

            <!-- Loading State -->
            @if (loading)
            {
                <div class="text-center py-5">
                    <MudProgressCircular Size="Size.Large" Indeterminate="true" />
                    <p class="mt-3 text-muted">Loading cars...</p>
                </div>
            }
            <!-- No Results -->
            else if (products == null || !products.Any())
            {
                <div class="text-center py-5">
                    <MudIcon Icon="@Icons.Material.Filled.SearchOff" Size="Size.Large" Class="text-muted mb-3" />
                    <h4 class="text-muted">No cars found</h4>
                    <p class="text-muted mb-4">Try adjusting your filters or search terms</p>
                    <MudButton Variant="Variant.Filled"
                             Color="Color.Primary"
                             StartIcon="@Icons.Material.Filled.Refresh"
                             @onclick="ClearFilters">
                        Clear Filters
                    </MudButton>
                </div>
            }
            <!-- Products Display -->
            else
            {
                @if (viewMode == ViewMode.Grid)
                {
                    <div class="row">
                        @foreach (var product in products)
                        {
                            <div class="col-xl-3 col-lg-4 col-md-6 mb-4">
                                <ProductCard Product="product"
                                           OnViewDetails="() => ViewProductDetails(product.Id)"
                                           OnAddToCart="() => AddToCart(product)"
                                           OnAddToWishlist="() => AddToWishlist(product)" />
                            </div>
                        }
                    </div>
                }
                else
                {
                    <div class="row">
                        <div class="col-12">
                            <MudTable Items="products"
                                    Hover="true"
                                    Striped="true"
                                    Dense="true"
                                    Breakpoint="Breakpoint.Sm">
                                <HeaderContent>
                                    <MudTh>Image</MudTh>
                                    <MudTh>Name</MudTh>
                                    <MudTh>Brand</MudTh>
                                    <MudTh>Price</MudTh>
                                    <MudTh>Year</MudTh>
                                    <MudTh>Condition</MudTh>
                                    <MudTh>Actions</MudTh>
                                </HeaderContent>
                                <RowTemplate>
                                    <MudTd>
                                        @if (!string.IsNullOrEmpty(context.ImageUrl))
                                        {
                                            <img src="@context.ImageUrl"
                                                 alt="@context.Name"
                                                 style="width: 60px; height: 45px; object-fit: cover; border-radius: 4px;" />
                                        }
                                        else
                                        {
                                            <MudIcon Icon="@Icons.Material.Filled.DirectionsCar"
                                                    Size="Size.Large"
                                                    Class="text-muted" />
                                        }
                                    </MudTd>
                                    <MudTd>
                                        <div>
                                            <strong>@context.Name</strong>
                                            @if (context.Category != null)
                                            {
                                                <br />
                                                <small class="text-muted">@context.Category.Name</small>
                                            }
                                        </div>
                                    </MudTd>
                                    <MudTd>@context.Brand</MudTd>
                                    <MudTd>
                                        <strong class="text-primary">@context.Price.ToString("N0") TND</strong>
                                   